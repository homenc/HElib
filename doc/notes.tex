
\documentclass[11pt]{article}

\newcommand{\FF}{\mathbf{F}}
\newcommand{\ZZ}{\mathbf{Z}}
\newcommand{\deq}{\mathrel{\mathop:}=}

\begin{document}
\sloppy

Assume $\gcd(m_1, m_2) = 1$, $p$ has order $d$ mod $m_1$,
and that $p^d \equiv 1 \pmod{m_2}$, so that $p$ also 
has order $d$ mod $m \deq m_1 m_2$.

Let $U$ be a set of integers whose images in $\ZZ/m_1\ZZ$
form a complete set of representatives for $(\ZZ/m_1\ZZ)^*/\langle p \rangle$.
Let $V$ be a set of integers whose images in $\ZZ/m_2\ZZ$
form a complete set of representatives for $(\ZZ/m_2\ZZ)^*$.

For each $i \in U$ and $j \in V$, there exists an integer $k$,
uniquely determined modulo $m$, such that 
\begin{equation}
\label{eq1}
\begin{array}{rcl}
m_2 k & \equiv & m_2 i \pmod{m} \\
m_1 k & \equiv & m_1 j \pmod{m} . \\
\end{array}
\end{equation}
Indeed, these two congrences are equivalent to
\begin{eqnarray*}
k & \equiv & i \pmod{m_1} \\
k & \equiv & j \pmod{m_2},
\end{eqnarray*}
and the claim follows from CRT.
Let $W$ be the set of integers $k$ that arise in this way.
The claim (verify) is that the image of $W$ in $(\ZZ/m\ZZ)^*$
forms a complete set of representatives fpr $(\ZZ/m\ZZ)^*/\langle p \rangle$.



Now consider the tower of rings $R = \FF_p[x_1]/(\Phi_{m_1}(x_1))$
and $S = R[x_2]/(\Phi_{m_2}(x_2))$.
We also have the ring $T = \FF_p[x]/(\Phi_m(x))$ and the $\FF_p$-algebra
isomorphism $\theta : S \rightarrow T$ that maps $x_1$ to $x^{m_2}$ and
$x_2$ to $x^{m_1}$.

Now we consider various evaluation maps.
Let $\xi$ be an element of order $m$ in $\FF_{p^d}$. 
The field elements $\xi^{m_2 i}$, for $i \in U$, each have order $m_1$,
and lie in distinct conjugacy classes relative to the Frobenius map
for $\FF_{p^d}/\FF_p$.
Therefore, as $i$ runs over $U$, the minimal polynomial of 
$\xi^{m_2 i}$ runs over the set of distinct irreducible factors
of $\Phi_{m_1}$.
The field elements $\xi^{m_1 j}$, for $j \in V$,
each have order $m_2$.
The field elements $\xi^k$, as $k$ runs over $W$,
each have order $m$, and lie in distinct conjugacy classes
relative to the Frobenius map for  $\FF_{p^d}/\FF_p$.
Therefore, as $k$ runs over $W$, the minimal polynomial of
$\xi^k$ runs over the set of distinct irreducible factors
of $\Phi_m$.

The evaluation map $\sigma_{i j} : S \rightarrow \FF_{p^d}$,
for $i \in U$ and $j \in V$, is defined so that
it sends $x_1 \mapsto \xi^{m_2 i}$ and $x_2 \mapsto \xi^{m_1 j}$.
The evaluation map $\tau_k : T \rightarrow \FF_{p^d}$,
for $k \in W$, sends $x \mapsto x^k$.
The claim (verify) is that for each $i \in U$ and $j \in V$,
if $k$ is the integer corresponding to $(i, j)$ as in (\ref{eq1}),
then
$\sigma_{i j} = \tau_k \circ \theta$;
that is, the evaluation map $\sigma_{i j}$ has the same effect as
mapping from $S$ to $T$ via the isomorphism $\theta$,
and then applying the evaluation map $\tau_k$.

\subsection*{Computation}

Now we want to consider the $\FF_p$-linear transformation 
that takes an element in $\alpha \in S$,
written on the natural $\FF_p$-basis for $S$,
and maps it to $\{\sigma_k(\theta(\alpha))\}_{k\in W}$.
In FHE-speak, this corresponds to the map that sends an object 
with its powerful-basis coordinates packed into slots to our
normal computational representation.

So let $\alpha = \sum_v x_2^v \sum_u a_{u v} x_1^u$,
where $u$ ranges over $0, \ldots, \phi(m_1)-1$,
and $v$ ranges over $0, \ldots, \phi(m_2)-1$. 
For each $v$, let $f_v(x_1) \deq \sum_u a_{u v} x_1^u$.
The first step is to compute 
$f_v(\xi^{m_2 i})$ for each $i \in U$  and $v = 0, \ldots, \phi(m_2)-1$.
So we are repeating the same multi-point evaluation map ``in parallel'',
and assuming slots are fully packed to begin with, we should be
able to do this 
with $\phi(m_1)/d$ inter-slot rotations and $d$ intra-slot (i.e., Frobenius)
rotations.

In more detail.
Assume we pack the coefficients of each $f_v$
into $\phi(m_1)/d$ slots.
To each such collection of $\phi(m_1)/d$ slots, 
we apply the same $\phi(m_1)/d \times \phi(m_1)/d$
block matrix, where each block is itself a $d \times d$ matrix.
Using generic SIMD linear-algebra techniques,
we should be able to do this in the stated complexity. 

Once we complete this first step, the second step is
to compute, in parallel, the multi-point evaluation maps that send
$x_2$ to to $\xi^{m_1 j}$ for $j \in V$.
These evaluation maps are $\FF_{p^d}$-linear maps,
and using generic SIMD linear-algebra techniques,
we should be able to do this using $\phi(m_2)$ inter-slot rotations.

\subsection*{Remarks}

The above assumes that we can efficiently implement the
required inter-slot rotations.
I'm fairly optimistic, but not entirely sure.
I would hope that in the worst case, we just need to do
a mask and two native rotations (well, this asumes we have
all the necessary key-switching matrices; if not they will cost more).

The above should trivially generalize to the base where
we have several moduli $m_1, \ldots, m_r$, where
$p$ has order $d$ mod $m_1$ and mod $m \deq m_1 \cdots m_r$.
For $p = 2$, it seems that one can get a pretty good variety of
$m$'s in this way, although it may not be enough for general
computations.

If our plaintext space is a power of $p$, I'm not sure what happens
to all of this theory. 
It might actually work with only small changes.
I know I've verified some of the linearized polynomial stuff
still works in this setting, although here, one cannot work
with an arbitrary field $\FF_{p^d}$,
but rather one must work explicitly with the ring of polynomials
modulo one of the irreducible factors of $\Phi_m$, or so it seems
to me.






\end{document}
